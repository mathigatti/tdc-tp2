\section{Resultados y análisis}

\subsection*{Universidad de São Paulo}

\subsubsection*{Recorrido en el Planisferio}

A continuación se puede ver de forma bastante clara como el paquete tuvo que pasar por estados unidos para luego ir a Europa Occidental hasta llegar finalmente a Rusia.

\begin{figure}[H]
  \centering
  \includegraphics[width=8.5cm]{figs/saopaulo.png}
  \caption{\normalfont Recorrido realizado por los paquetes durante la ejecución de traceroute al intentar alcanzar el sitio XXXXXXXXXXXXXXXXXXXXXXXXXX}
\end{figure}

\subsubsection*{RTT entre saltos}

A continuación intentamos ver los timepos entre saltos. Para esto calculamos la media de cada de los RTTs obtenidos de cada TTL para reducir a solo un valor las mediciones obtenidas por cada TTL. 

Seguido de esto realizamos dos experimentos. Primero simplemente restamos los RTTs medios entre si como se puede ver en el primer gráfico, luego llevamos el experimento un paso mas allá normalizando con z-score los RTTs para eliminar cualquier tipo de deformación en las dimensiones del gráfico.

\begin{figure}[H]
  \centering
  \includegraphics[width=8.5cm]{figs/traceroute-saopaulo.pdf}
  \caption{\normalfont RTT entre saltos (antes y después de normalizar respectivamente) para el sitio XXXXXXXXXXXXXXXXXXXXXX}
\end{figure}

Como se puede observar ... COMPLETAR.


\subsection*{Universidad de Sidney}

\subsubsection*{Recorrido en el Planisferio}

A continuación se puede ver de forma bastante clara como el paquete tuvo que pasar por estados unidos para luego ir a Europa Occidental hasta llegar finalmente a Rusia.

\begin{figure}[H]
  \centering
  \includegraphics[width=8.5cm]{figs/sidney.png}
  \caption{\normalfont Recorrido realizado por los paquetes durante la ejecución de traceroute al intentar alcanzar el sitio sydney.edu.au}
\end{figure}

\subsubsection*{RTT entre saltos}

A continuación intentamos ver los timepos entre saltos. Para esto calculamos la media de cada de los RTTs obtenidos de cada TTL para reducir a solo un valor las mediciones obtenidas por cada TTL. 

Seguido de esto realizamos dos experimentos. Primero simplemente restamos los RTTs medios entre si como se puede ver en el primer gráfico, luego llevamos el experimento un paso mas allá normalizando con z-score los RTTs para eliminar cualquier tipo de deformación en las dimensiones del gráfico.

\begin{figure}[H]
  \centering
  \includegraphics[width=8.5cm]{figs/traceroute-sidney.pdf}
  \caption{\normalfont RTT entre saltos (antes y después de normalizar respectivamente) para el sitio sydney.edu.au}
\end{figure}

Como se puede observar ... COMPLETAR.


\subsection*{Universidad de Moscú}

\subsubsection*{Recorrido en el Planisferio}

A continuación se puede ver de forma bastante clara como el paquete tuvo que pasar por estados unidos para luego ir a Europa Occidental hasta llegar finalmente a Rusia.

\begin{figure}[H]
  \centering
  \includegraphics[width=8.5cm]{figs/moscow.png}
  \caption{\normalfont Recorrido realizado por los paquetes durante la ejecución de traceroute al intentar alcanzar el sitio www.msu.com}
\end{figure}

\subsubsection*{RTT entre saltos}

A continuación intentamos ver los timepos entre saltos. Para esto calculamos la media de cada de los RTTs obtenidos de cada TTL para reducir a solo un valor las mediciones obtenidas por cada TTL. 

Seguido de esto realizamos dos experimentos. Primero simplemente restamos los RTTs medios entre si como se puede ver en el primer gráfico, luego llevamos el experimento un paso mas allá normalizando con z-score los RTTs para eliminar cualquier tipo de deformación en las dimensiones del gráfico.

\begin{figure}[H]
  \centering
  \includegraphics[width=8.5cm]{figs/traceroute-moscow.pdf}
  \caption{\normalfont RTT entre saltos (antes y después de normalizar respectivamente) para el sitio www.msu.com}
\end{figure}

Como se puede observar ... COMPLETAR.