\section{Introducción}

Este trabajo gira en torno al funcionamiento y aplicación de traceroute, una herramienta que permite analizar y seguir la traza de los distintos nodos por los que pasa un paquete cuando intenta alcanzar cierto host destino. Para conseguir el mejor entendimiento de la herramienta, realizaremos nuestra propia implementación.

Utilizando métricas como la latencia de red en conjunto con la ubicación de las IPs de los hops intentaremos aprender sobre los caminos que realizan los paquetes de internet y los tiempos que estos manejan. Finalmente implementaremos y evaluaremos un método para detectar automáticamente enlaces continentales a partir de un análisis estadístico de los Round Trip Times (RTT).

A lo largo de este informe describiremos los detalles de la implementación realizada, los resultados obtenidos y sus limitaciones.