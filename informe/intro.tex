\section{Introducción}

Este trabajo gira en torno al funcionamiento y aplicación de traceroute, una herramienta que permite analizar y seguir la pista de los distintos puntos por los que pasa un paquete cuando intenta alcanzar cierto host destino.

Utilizando métricas como la latencia de red en conjunto con la ubicación de las IPs de los hops intentaremos aprender sobre los caminos que realizan los paquetes de internet y los tiempos que estos manejan. Finalmente desarrollaremos y examinaremos técnicas para detectar de forma automática enlaces intercontinentales.

A lo largo de este informe describiremos los detalles de la implementación realizada, los resultados obtenidos y sus limitaciones.