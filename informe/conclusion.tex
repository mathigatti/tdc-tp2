\section{Discusión}

Recapitulando, de lo dicho hasta aquí nos llevamos las siguientes conclusiones:

\begin{itemize}
	\item Vimos, con el caso de Rusia, que este modelo es muy sensible a anomalías como los caminos asimétricos (que sospechamos que fue uno de los problemas), o las congestiones en la red (otra suposición). Este caso también ilustra una situación que no puede mejorarse con una valor de corte arbitrario (distinto de $\tau$) dado que los verdades saltos continentales tienen un RTT diferencial menor a los falsos.
	\item Aún en el caso de Australia, donde no parecieron verse anomalías grandes, el modelo tuvo problemas, infiriendo falsos enlaces continentales. El modelo parece demasiado simple para lidiar con la arbitrariedad de qué es y qué no un continente. Es notable que, por cómo funciona el método de Cimbala (ver un outlier a la vez, y sacarlo), aunque se detecten en forma correcta los enlaces continentales primero, luego quedan un set de puntos intracontinentales donde una gran distancia puede resultar un outlier (que no lo era si considerábamos el set completo). Esto hace que un caso como el de Australia, pueda terminar reduciéndose a varios como el de Brasil, encontrando falsos positivos. Posibles soluciones a esto serían tener siempre en consideración el \emph{big picture} de todos los puntos como parte del score. También podría devolverse la lista de outliers con un nivel de confianza sobre la posibilidad de que sean enlaces continentales o no. En definitiva, la influencia de la longitud de la ruta para el método actual es menor a lo que se esperaría.
	\item Entre un 10 y un 30 por ciento de los hops ignoraron la respuesta por time exceeded, esto nos llamó la atención ya que parece ser un número bastante alto, aunque investigando un poco descubrimos que este fenómeno no es para nada extraño. Las razones más comunes por lo que esto suele ocurrir parecen ser la configuración del hop para omitir la respuesta a estos mensajes y el bloqueo de paquetes desde el firewall según indica la documentación de traceroute \footnote{http://web.mit.edu/freebsd/head/contrib/traceroute/traceroute.c} y el artículo provisto por la cátedra \footnote{https://www.net.in.tum.de/fileadmin/TUM/NET/NET-2012-08-1/NET-2012-08-1\_02.pdf}.

\end{itemize}

En definitiva, notamos que este modelo tiene muchas oportunidades de mejora, y no es lo suficientemente robusto como para poder ser usado seriamente como un detector de enlaces intercontinentales. De hecho, salvo que se cuente con cierta metadata sobre el estado de la red, parece imposible tener un buen predictor, considerando la gran variedad de situaciones que se dan actualmente en las redes y que pueden meter ruido, desde congestiones hasta anomalías debidas a las topologías o los protocolos heterogéneos.
