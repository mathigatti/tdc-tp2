\section{Métodos}

\subsection*{Herramientas}
Para la implementación de traceroute utilizamos el código provisto por la catedra al cual le realizamos ciertas modificaciones para poder detectar anomalías y guardar los datos obtenidos de forma más cómoda.

\subsection*{Detección de Anomalías}

Como se describe en Cimbala\footnote{http://www.net.in.tum.de/fileadmin/TUM/NET/NET-2012-08-1/
NET-2012-08-1\_02.pdf} aplicamos un cálculo el cual compara los datos normalizados con z-score con una valor $\tau$ derivado del t-student con un $\alpha$ de 0.05 lo cual modela un intervalo de confianza el cual descarta los 0.025 percentiles.


\subsubsection*{Capturas}
Se corrió traceroute sobre 3 universidades distintas con un ttl de 30 y 40 queries. Con el objetivo de lograr contrastar elegimos universidades muy lejanas y muy cercanas. A continuación describimos brevemente a cada una.

\begin{itemize}
	\item Universidad de São Paulo (www5.usp.br) esta será la universidad mas cercana, ubicada en el mismo continente, por lo cual esperamos que no haya ningún enlace intercontinental.
	\item Universidad de Sidney (www.sydney.edu.au) escogimos esta universidad ya que nos surgió la duda de si existirá algún enlace intercontinental directo entre oceanía y america o tendrá que pasar europa resultando en varios enlaces.
	\item Universidad de Moscú (www.msu.ru) al estar ubicada en un punto tan alejado de nosotros estabamos seguros de que iba a haber algún salto intercontinental y quizás más.
\end{itemize}